\section{Executive Summary}

In the past few years the web development industry has matured
greatly.  An increased emphasis on professionalism, adherence to
best-practices, industry standardization, and steadily expanding
Internet use, combined with advances in the capabilities of popular
web browsers has fueled a new era for web developers.

\subsection{The Problem}

Central to this new professionalism is the importance placed on the
application's responsiveness, dynamic content, making use of web
services APIs asynchronously, AJAX, and all manner of clientside
technologies. At the same time, however, the overarching requirements
of search engine optimization and accessibility can not be neglected
by any serious developer.

Correspondingly, as the capabilities, size, and scope of web
applications grows, organization and architecture issues become
critical. Unfortunately, many of the standard best-practices that are
successfully used by the software industry as a whole are not entirely
feasible in the domain of the web application. The particular
difficulties of web development (i.e. the dependence on serverside
scripting to generate HTML output, the stateless nature of HTTP, and
the fact that javascript-disabled browsers must still be supported, to
name a few), have made a coherent, object-oriented, modular web
development architecture a rather elusive pie-in-the-sky kind of
ideal.

That is not to say that there are no good web development frameworks
out there, because there are many. But in terms of overall
organization, serverside scripting involves limitations that are
significant and should be noted.

\subsection{What's Different About Golf?}

The Golf stack attacks this problem head-on. The basic enabling
technology it provides is a fully-functioning javascript proxy
capability. The Golf application server can automatically and 
transparently detect when a browser does not have javascript, or
if javascript is disabled, and provide the complete page in pure
HTML, with special hooks to proxy events. This is covered in
detail in Section \ref{sec:howitworks}.

This javascript proxying capability is the foundation that frees
the developer to think about web applications in a whole new way.
