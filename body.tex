\section{Executive Summary}

The Golf web application developent stack was concieved in response
to certain well-known problems with current web application
architectures:

\begin{itemize}
\end{itemize}

\section{Introduction}

In the past few years the web development industry has matured
greatly.  An increased emphasis on professionalism, adherence to
best-practices, industry standardization, and steadily expanding
Internet use, combined with advances in the capabilities of popular
web browsers has fueled a new era for web developers.

\subsection{The Problem}

Central to this new professionalism is the importance placed on the
application's responsiveness, dynamic content, making use of web
services APIs asynchronously, AJAX, and all manner of clientside
technologies. At the same time, however, the overarching requirements
of search engine optimization and accessibility can not be neglected
by any serious developer.

Correspondingly, as the capabilities, size, and scope of web
applications grows, organization and architecture issues become
critical. Unfortunately, many of the standard best-practices that are
successfully used by the software industry as a whole are not entirely
feasible in the domain of the web application. The particular
difficulties of web development (i.e. the dependence on serverside
scripting to generate HTML output, the stateless nature of HTTP, and
the fact that javascript-disabled browsers must still be supported, to
name a few), have made a coherent, object-oriented, modular web
development architecture a rather elusive pie-in-the-sky kind of
ideal.

That is not to say that there are no good web development frameworks
out there, because there are many. But in terms of overall
organization, serverside scripting involves limitations that are
significant and should be noted.

With a completely clientside presentation layer, however, the boundary
between server and client is pushed way back. The server would have
absolutely no responsibility for the presentation layer, whatsoever.
Under this scheme the server would only be required to serve up the
static javascripts, and provide some kind of web services backend for
data and business logic. Everything else would be conducted in the
client's browser using javascript.

\subsection{What's Different About Golf?}

The Golf stack attacks this problem head-on. The basic enabling
technology it provides is a fully-functioning javascript proxy
capability. The Golf application server can automatically and
transparently detect when a browser does not have javascript, or if
javascript is disabled, and provide the complete page in pure HTML,
with special hooks to proxy events. This is covered in detail in
Section \ref{sec:howitworks}.

This javascript proxying capability is the foundation that frees the
developer to think about web applications in a whole new way.  Once
the dependence on serverside scripting (to provide the main site
functionality, according to principles of progressive-enhancement) is
removed, the possibility and power of true clientside architectures
become attainable.

Golf provides precisely such a framework. Starting from an empty
document and the Golf javascript clientside MVC framework, the
application's javascript controller instantiates views made up of Golf
"components" and connects them up to clientside javascript REST client
models. The models use AJAX to interact with the business logic and
data layers via an independent RESTful web service running on the
server. At some point the controller will insert the Golf "component"
into the DOM using jQuery, just as a normal DOM element would be.

These components are little modules, encapsulating DOM structure, CSS
styling, static resources, and dynamic javascript behaviors.
Successful encapsulation makes an object-oriented approach feasible
for web applications.

\subsection{Incidental Benefits}

Since Golf is a fully clientside framework, only the javascripts
themselves are served to the client. No serverside processing is
needed in a Golf application. This means that the server simply
spends all day just delivering static content. Of course, static
content can be offloaded onto a content delivery network, such as
Amazon's CloudFront, for example.

